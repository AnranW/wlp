%!TEX root = ../main-anran-ma.tex 
% so I can build in this tex file too. 
%************************************************
\chapter{A Proof System}\label{ch:system} % $\mathbb{ZNR}$
%************************************************

We are interested in studying the \define{necessary liberal precondition}, a weakening of the weakest liberal precondition: 
$$wlp.C.F\implies G$$
The weaker $G$ can contain various preconditions: on the one hand, $G$ can be so general that it is satisfied by any program state; on the other hand, a $G$ that is barely weaker than $wlp.C.F$ is also not much different from the latter. 
Alternatively, $G$ can also contain all kinds of preconditions that starting from it, any postcondition is reachable. 
One thing we are certain about, though, is that a program with an original state satisfying $\neg G$ will terminate, and the final state can satisfy $\neg F$: 
\begin{align*}
wlp.C.F\implies G & = \neg G \implies \neg wlp.C.F \\
	& = \neg G \implies wp.C.\neg F 
	\hspace{0.3\textwidth} | \ \todo{insert theorem: wlp and wp are conjugates} 
\end{align*}
In the upcoming sections, we first discuss various forms that the necessary liberal precondition can take and try to identify a $G$ that is most characteristic. 
We proceed then to propose a proof system stemming from the necessary liberal precondition and show its usefulness using an example. \todo{replace with concrete example} 

\section{A Liberal Precondition Weaker Than the Weakest Liberal Precondition }
In \autoref{sec:wlp} we defined the weakest liberal precondition and state that it characterizes all the preconditions under whose control the program either \imptt{diverges} or \imptt{will} terminate in a state satisfying $F$. 
We are certain to use ``will'' instead of ``can'', because we view the non-determinism as demonic, so the behavior of wlp can be depicted by \autoref{subfig:wlpd}. 
We can categorize the executions of the program in four ways: 
\begin{enumerate}
	\item the dashed arrow means non-terminating executions; 
	\item the black arrows are executions starting from an initial state satisfying $wlp.C.F$ and only terminating in final states satisfying $F$; 
	\item the green arrows are the executions starting from an initial state satisfying $\neg wlp.C.F$ but can terminate in states either satisfying $F$ or satisfying $\neg F$;
	\item the red arrow represents executions starting from an initial state satisfying $\neg wlp.C.F$ and only terminating in final states satisfying $\neg F$. 
\end{enumerate}

If we were to weaken the precondition, it can happen in various ways as shown in \autoref{subfig:wlp-g-g}{\color{RoyalBlue}-9}. 
We argue that $G$ is most characteristic, when it takes the form as in \autoref{subfig:wlp-g-gg}, because under its control, the program always \imptt{can} reach a final state satisfying $F$ if it terminates, while with an initial state satisfying $\neg G$, the program is \imptt{will} terminate satisfying $\neg F$. 
This behavior is exactly the behavior of wlp, if we were to regard the non-deterministic choice as angelic, as hinted by the similarities between \autoref{subfig:wlp-g-gg} and \autoref{subfig:wp-angelic}. 

\begin{figure}[ht!]\centering
	\subfloat[Weakest liberal precondition (demonic nondeterminism) \label{subfig:wlpd}]{
		\includegraphics[width=0.4\textwidth]{image/wlp-g/wlpd.eps}
	}
	\hfill

	\subfloat[Precondition $G$ with $wlp.C.F\implies G$ and $G$ contains some green arrows \label{subfig:wlp-g-g}]{
		\includegraphics[width=0.45\textwidth]{image/wlp-g/wlp-g-g.eps}
	}
	\hfill
	\subfloat[Precondition $G$ with $wlp.C.F\implies G$ and $G$ contains all the green arrows \label{subfig:wlp-g-gg}]{
		\includegraphics[width=0.45\textwidth]{image/wlp-g/wlp-g-gg.eps}
	}

	\subfloat[Precondition $G$ with $wlp.C.F\implies G$ and $G$ contains some red arrows \label{subfig:wlp-g-r}]{
		\includegraphics[width=0.45\textwidth]{image/wlp-g/wlp-g-r.eps}
	}
	\hfill
	\subfloat[Precondition $G$ with $wlp.C.F\implies G$ and $G$ contains all the red arrows \label{subfig:wlp-g-rr}]{
		\includegraphics[width=0.45\textwidth]{image/wlp-g/wlp-g-rr.eps}
	}
\caption{Case Distinction of Preconditions Weaker Than wlp (Part 1) }
\label{fig:wlp-g-1}
\end{figure}

\begin{figure}[ht!]\centering
	\ContinuedFloat
	\subfloat[Precondition $G$ with $wlp.C.F\implies G$ and $G$ contains some green arrows and some red arrows \label{subfig:wlp-g-gr}]{
		\includegraphics[width=0.45\textwidth]{image/wlp-g/wlp-g-gr.eps}
	}
	\hfill
	\subfloat[Precondition $G$ with $wlp.C.F\implies G$ and $G$ contains all the green arrows and some red arrows \label{subfig:wlp-g-ggr}]{
		\includegraphics[width=0.45\textwidth]{image/wlp-g/wlp-g-ggr.eps}
	}

	\subfloat[Precondition $G$ with $wlp.C.F\implies G$ and $G$ contains some green arrows and all the red arrows \label{subfig:wlp-g-grr}]{
		\includegraphics[width=0.45\textwidth]{image/wlp-g/wlp-g-grr.eps}
	}
	\hfill
	\subfloat[Precondition $G$ with $wlp.C.F\implies G$ and $G$ contains all the arrows \label{subfig:wlp-g-ggrr}]{
		\includegraphics[width=0.45\textwidth]{image/wlp-g/wlp-g-ggrr.eps}
	}
\caption{Case Distinction of Preconditions Weaker Than wlp (Part 2) }
\label{fig:wlp-g-2}
\end{figure}

Luckily, we can find statements using wlp and sp that captures this specific $G$, hence giving us a way to express wlp with angelic non-determinism (denoted by $wlp_a$) without having to define it: 
\begin{lemma}[Angelic wlp implies G]
{If }$wlp.C.F\implies G${ and }$sp.C.\neg G \implies \neg F${ then }$wlp_a.C.F \implies G$. 
\end{lemma}

\begin{lemmabk}[G implies angelic wlp]
\tab[5mm]{If }$wlp.C.F\implies G${ and }$(P\implies G) \implies \neg(sp.C.P \implies \neg F)${ then }$G \implies wlp_a.C.F$. 
\end{lemmabk}

\begin{corollarybk} [G equivalent to angelic wlp]
{If }$wlp.C.F\implies G${ and }$sp.C.\neg G \implies \neg F$ 
{ and }$(P\implies G) \implies \neg(sp.C.P \implies \neg F)$
{ then }$G = wlp_a.C.F$. 
\end{corollarybk}


% bold version of the above lemmas
% \begin{lemma}[Angelic wlp implies G]
% \textbf{If }$wlp.C.F\implies G$\textbf{ and }$sp.C.\neg G \implies \neg F$\textbf{ then }$wlp_a.C.F \implies G$. 
% \end{lemma}

% \begin{lemma}
% \textbf{If }$wlp.C.F\implies G$\textbf{ and }$(P\implies G) \implies \neg(sp.C.P \implies \neg F)$\textbf{ then }$G \implies wlp_a.C.F$. 
% \end{lemma}

% \begin{corollary}
% \textbf{If }$wlp.C.F\implies G$\textbf{ and }$sp.C.\neg G \implies \neg F$ 
% \textbf{ and }$(P\implies G) \implies \neg(sp.C.P \implies \neg F)$
% \textbf{ then }$G = wlp_a.C.F$. 
% \end{corollary}














\newpage
\section{A Proof System}





%*****************************************
%*****************************************
%*****************************************
%*****************************************
%*****************************************
