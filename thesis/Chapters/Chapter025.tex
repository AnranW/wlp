%!TEX root = ../main-anran-ma.tex 
% so I can build in this tex file too. 
%************************************************
% \setcounter{theorem}{6}
\chapter{Preconditions, Postconditions, and Their Approximations}\label{ch:appr} % $\mathbb{ZNR}$
In this chapter, we give an overview of precondition and postcondition transformers presented in the previous chapter, establishing relations between all of them using termination, conjugation, reachability, and Galois connections. 
This reveals that the necessary liberal precondition triple that we investigate in this thesis is still an open question in the sense that it is not yet well-researched to our knowledge, and that it is not automatically related to another triple by the aforementioned relations. 
Furthermore, the methods mentioned in this chapter used to study other triples prove helpful while investigating necessary liberal preconditions.   

% \section{Preconditions and Postconditions}\label{sec:pre-post}
\section{Precondition Transformers}
\begin{figure}[t]\centering
	\subfloat[Weakest precondition (angelic non-determinism)\label{subfig:wpa}]{
		\includesvg[width=0.4\textwidth]{image/wpa.svg}}
	\hfill
	\subfloat[Weakest precondition (demonic non-determinism)\label{subfig:wpd}]{
		\includesvg[width=0.4\textwidth]{image/wpd.svg}}

	\subfloat[Weakest liberal precondition (angelic non-determinism)\label{subfig:wlpa}]{
		\includesvg[width=0.4\textwidth]{image/wlpa.svg}}
	\hfill
	\subfloat[Weakest liberal precondition (demonic non-determinism)\label{subfig:wlpd}]{
		\includesvg[width=0.4\textwidth]{image/wlpd.svg}}
	\caption{wp, wlp with Angelic and Demonic Non-determinism}
	\label{fig:wp-family}
\end{figure}

\autoref{fig:wp-family} shows the impact of the interpretation of non-determinism on the semantics of the wp and wlp transformers. 
The subscripts $_a$ and $_d$ correspond to angelic and demonic resolution of non-determinism mentioned in \autoref{sec:wp-nondet}, respectively. 
The notion of angelic and demonic is reflected by the placement of green dots and arrows in \autoref{fig:wp-family}.

In general, the rectangles denote the set of all states $\S$. 
$C$ is the program that we are investigating, all the arrows are executions of program $C$. 
When a rectangle is divided by a line into multiple parts, it means that the states are categorized into different types. 
For example, the section  labelled with $wlp.C.F$ denotes all the states that satisfy $wlp.C.F$. 
The dashed arrow starting from inside a rectangle, ending outside any rectangle denotes executions that do not terminate. 
The arrows that share the same starting point denote different executions starting from the same initial state, and those that share an ending point denote executions that end in the same final state. 

With angelic non-determinism, we consider the non-deterministic choice to resolve in our favor, which means that if one of the possible paths from this non-deterministic choice can lead to the desired postcondition, we regard the precondition as a valid precondition. 
Graphically, the green arrows in \autoref{subfig:wpa} and \autoref{subfig:wlpa} are included in the weakest (liberal) precondition when we regard the non-determinism as angelic. 
Formally, the decision to look at the non-deterministic choice as angelic is reflected by the use of disjoints in its definition, for example, 
$$wp.C.(\{C_1\}\square \{C_2\}) = wp.C_1.F\vee wp.C_2.F$$ for a postcondition $F$ in \autoref{tab:wp-wlp}.

On the other hand, if the non-deterministic choice is resolved as demonic, we are pessimistic over the result of the non-deterministic choice: we assume it will not resolve to our favor, hence we are only confident to include a condition into the valid precondition set if all possible paths end in a final state satisfying our desired postcondition upon termination. 
Graphically, the green arrows are not included in $wp_d$ and $wlp_d$ in \autoref{subfig:wpd} and \autoref{subfig:wlpd}. 
The definition is then using conjuncts instead of disjuncts, like 
$$wlp.C.(\{C_1\}\square \{C_2\}) = wlp.C_1.F\wedge wlp.C_2.F$$ for a postcondition $F$ in \autoref{tab:wp-wlp}.

Note that we keep the figures in \autoref{fig:wp-family} clear and compact but still keep the essence of the predicate transformers. 
A more detailed version can be found in \autoref{fig:wp-wlp-extended}.  

The choice is somewhat artificial, depending on what properties the author wishes for their transformers.
For example, Dijkstra and Scholten choose demonic non-determinism for wp and wlp transformers, acquiring a property by definition~\cite{dijkstra90}: 
$$wp_d.C.F = wlp_d.C.F \wedge wp_d.C.true $$
This means that by finding the weakest liberal precondition and proving termination, one conveniently finds the weakest precondition of the same program and postcondition. 
Graphically, one needs only to re-place the dashed arrows indicating non-termination to get from \autoref{subfig:wpa} to \autoref{subfig:wlpa} and vice versa, or from  \autoref{subfig:wpd} to \autoref{subfig:wlpd} and vice versa. 

This is especially useful when trying to underapproximate the weakest precondition, %TODO: add reference?
or in other words, finding valid Hoare Triple, considering that \hoare{G}{C}{F} is a valid Hoare Triple (with regard to total correctness, meaning that we supplement the original axioms~\cite{hoare69} with rules to prove termination~\cite{manna74}) if and only if $G\implies wp.C.F$.
% The rule also applies if one chooses both angelic 

Conversely, if one were to choose different views on non-determinism with wp and wlp transformers, for example angelic resolution for wp and demonic resolution for wlp, we end up with a different property that wp and wlp are each other's \define{conjugates}~\cite{dijkstra76,zhang22}: 
$$ wp_a.C.F = \neg wlp_d.C.\neg F $$ 
This indicates a duality between $wp_a$ and $wlp_d$ in the sense that by defining one transformer, we can arrive at the other by negation. 
Graphically, we can ``flip'' \autoref{subfig:wpd} to arrive at \autoref{subfig:wlpa} and vice versa, or ``flip'' \autoref{subfig:wpa} to arrive at \autoref{subfig:wlpd} and vice versa, like in \autoref{fig:flip-wp-wlp}. 
\begin{figure}[ht]
	\centering
	\includesvg[width=.9\linewidth]{image/flip-wp-wlp.svg}
	\caption{wp$_a$ and wlp$_d$ Are Conjugates}
	\label{fig:flip-wp-wlp}
\end{figure}

These relations can be summed up as in \autoref{fig:wp-family-relation}. 
The choice whether to resolve the non-determinism angelically or demonically for wp and wlp depends then on which of the relations the author prefers. 
\begin{figure}[ht]
	\centering
	\includesvg[width=.5\linewidth]{image/wp-family-relation.svg}
	\caption{Relations in wp-Family}
	\label{fig:wp-family-relation}
\end{figure}

In this thesis, we prefer wp with angelic non-determinism and wlp with demonic non-determinism, because this choice gifts us Galois connection between the precondition transformers and the postcondition transformers~\cite{zhang22}, which we will illustrate in \autoref{sec:literature}. 


\section{Postcondition Transformers}
The strongest postcondition was originally discussed as a side-product of the wp and wlp transformers by Dijkstra and Scholten~\cite{dijkstra90}. 
$sp.C.G$ describes ``those final states for which there exists a computation under control of $C$ that belongs to the class \textit{initially G}'', but it has found more important utilities. 
For example, de Vries and Koutavas~\cite{vries11} use sp to reason about randomized algorithms, although they called it \define{weakest postconditions}. 
O'Hearn~\cite{ohearn2020IncorrectnessLogic} uses sp to define \define{incorrectness logic} by underapproximating it, building the theoretical foundation for a new method of identifying errors in programs. 

Instead of reasoning about termination and final states upon termination as precondition transformers do, the postcondition transformers reason about reachability and initial states that can reach the desired postconditions. 
The choice of angelic and demonic non-determinism also plays a rule while determining the semantics of the postcondition transformers, as shown in \autoref{fig:sp-family}. 

\begin{figure}[ht!]\centering
	\subfloat[Strongest postcondition (angelic non-determinism)\label{subfig:spa}]{
		\includesvg[width=0.4\textwidth]{image/spa.svg}}
	\hfill
	\subfloat[Strongest postcondition (demonic non-determinism)\label{subfig:spd}]{
		\includesvg[width=0.4\textwidth]{image/spd.svg}}

	\subfloat[Strongest liberal postcondition (angelic non-determinism)\label{subfig:slpa}]{
		\includesvg[width=0.4\textwidth]{image/slpa.svg}}
	\hfill
	\subfloat[Strongest liberal postcondition (demonic non-determinism)\label{subfig:slpd}]{
		\includesvg[width=0.4\textwidth]{image/slpd.svg}}
	\caption{sp, slp with Angelic and Demonic Non-determinism}
	\label{fig:sp-family}
\end{figure}

Note that while sp is well-researched, slp is less investigated to the best of our knowledge. 
For example, Wulandari and Plump~\cite{wulandari2020VerifyingGraphPrograms} defined and proved soundness for slp, but only for loop-free programs. 
Li et al.~\cite{li2011NonlinearMathematicsUncertainty} proposed the definition of slp including loops, but did not include soundness proof, while Zhang and Kaminski~\cite{zhang22-full} provided a definition of slp and soundness proof using collecting semantics. 
% Even so, the semantics of slp is rather undisputed: slp should be sp joint with all unreachable postconditions. 

Following the same principles as the ones for precondition transformers, we can deduce a similar relation between the postcondition transformers with different resolution for non-determinism, as seen in \autoref{fig:sp-family-relation}. 
The main difference is that the arrows pointing up are not labelled with termination, but reachability. 

\begin{figure}[t]
	\centering
	\includesvg[width=.5\linewidth]{image/sp-family-relation.svg}
	\caption{Relations in sp-Family}
	\label{fig:sp-family-relation}
\end{figure}

\section{Approximation Triples in Literature}\label{sec:literature} 
Now that we established relations between the precondition transformers and between the postcondition transformers, we can attempt to also establish relations that link the pre- and postcondition transformers together, this is where the Galois connection comes into play: 

\begin{theorem}{thm:galois-wp}~\cite{zhang22}[Galois connection between wp and slp]
	$$wp_a.C.F \implies G \text{ if and only if }F\implies slp_d.C.G$$
\end{theorem}

\begin{theorem}{thm:galois-wlp}~\cite{zhang22}[Galois connection between wlp and sp]
	$$G \implies wlp_d.C.F \text{ if and only if } sp_a.C.G\implies F$$
\end{theorem}

Supplementing the Galois connection specified in \thm{thm:galois-wp} and \thm{thm:galois-wlp}, we arrive at \autoref{fig:wp-sp-relation}. 
It is now obvious that we chose to resolve non-determinism exactly like the colored ones in \autoref{fig:wp-sp-relation}, so that we can conveniently connect the pre- and postcondition transformers. 

\begin{figure}[ht]
	\centering
	\includesvg[width=\linewidth]{image/wp-sp-relation-h.svg}
	\caption{Connecting wp-Family and sp-Family}
	\label{fig:wp-sp-relation}
\end{figure}

The advantage of the design choice of wp and sp with angelic non-determinism as well as wlp and slp with demonic non-determinism shows itself now. 
By choosing the formalism that are colored in red and blue in \autoref{fig:wp-sp-relation}, all the under- and overapproximations over the predicate transformers are related well together, as we depict in \autoref{fig:triples}.
\begin{figure}[h]
	\centering
	\includesvg[width=\linewidth]{image/existing-triples.svg}
	\caption{Connecting Overapproximations and Underapproximations}
	\label{fig:triples}
\end{figure}

Some of the implications in \autoref{fig:wp-sp-relation} are well-studied by now, as we will see in the upcoming paragraphs. 
Since the view on angelic or demonic non-determinism varies depending on the authors' choice, and do not make fundamental differences on their results, we will mention it in the description, but drop the subscripts $_a$ and $_d$ in the remainder of the section. 

\paragraph{Paritial Correctness}
The blue implication $$G\implies wlp.C.F$$ corresponds to a Hoare triple~\cite{hoare69} with respect to partial correctness: \hoare G C F is valid if and only if $G\implies wlp.C.F$~\cite{gordon2010ForwardHoare,nipkow2002isabelle}. 
This means that if the program $C$ starts in a state satisfying $G$, then the execution must end in a final state satisfying $F$, if it terminates. 
In terms of approximation, we can say that the precondition in the valid Hoare triple is an underapproximation of the weakest precondition. 

Hoare logic does not originally contain rules for non-deterministic choice, but it can be added without much difficulty, for example with demonic non-determinism~\cite{nipkow2002HoareLogicsIsabelle}: 
\begin{center}
	\begin{prooftree}
		\Hypo{$\hoare G {C_1} F,\ \  \hoare G {C_2} F$}
		\infer1{$\hoare G {C_1\square C_2}F$}
	\end{prooftree}
\end{center}
As we can see in \autoref{fig:wp-sp-relation}, the blue implications satisfy the if-and-only-if relation. 
This means that an overapproximation of the strongest postcondition $$sp.C.G\implies F$$ also guarantees partial correctness, aka \hoare{G}C F is a valid Hoare triple. 

\paragraph{Total Correctness}
Manna and Pnueli~\cite{manna74} supplimented Hoare triple with rules to prove termination, making the extended proof system valid for total correctness: 
\begin{center}
	\begin{prooftree}
		\Hypo{$\hoare{P\wedge\varphi\wedge v\in\N\wedge v=n\ \ }{C}{\ \ P\wedge\phi\wedge v\in\N\wedge v<n}$}
		\infer1{$\hoare {P\wedge v\in\N\ \ } {while\ (\varphi)\ do\ C} {\ \ \neg\varphi\wedge P\wedge t\in\N}$}
	\end{prooftree}
\end{center}
$P$ is the so-called \define{invariant}, a condition that stays unchanged before and after the program execution. 
$v$ in this case is the \define{variant}, it is modified by the execution of the program. 
Specifically, $v$ reduces with each execution of the sub-program $C$ inside the loop body. 
Since $v$ is a natural number, it is never negative, i.e. bound from the bottom. 
Consequently, the execution of the loop terminates at some point, and the loop body $C$ can be executed maximum $n$ times. 
This version of Hoare logic proves total correctness (termination and correctness), and it corresponds to the implication labelled with ``total correctness'' in \autoref{fig:wp-sp-relation}: 
$$G\implies wp.C.F$$

\paragraph{Total Incorrectness}
In the correctness realm, we are concerned with termination and guaranteeing that only good things can happen in the end. 
Changing the perspective to focus on reachability of erroneous final states, we end up in the incorrectness realm. 
The implication marked with ``total incorrectness'' in \autoref{fig:wp-sp-relation} was studied by O'Hearn~\cite{ohearn2020IncorrectnessLogic} under the name ``incorrectness logic'', and by de Vries and Koutavas~\cite{vries11} with the name "reverse Hoare logic": 
$$F\implies sp.C.G$$
With this triple, we can prove the reachability of states described by $F$ starting from the initial condition satisfying $G$. 
We can view it as an indication that the program $C$ was not constructed correctly, and bugs described by $F$ are reachable~\cite{ohearn2020IncorrectnessLogic}; or use it to prove that the final states in $F$ are reachable via randomized programs~\cite{vries11}. 

In reverse Hoare logic and incorrectness logic, the resolution of non-determinism is considered as angelic, same as how Dijkstra and Scholten view the resolution of non-determinism with sp~\cite{dijkstra90}.  
For example, O'Hearn defines the rule for non-deterministic choice as follows ($\epsilon$ is either $ok$ or $er$, an indicator whether the final condition is reached normally or erroneously): 
\begin{center}
	\begin{prooftree}
		\Hypo{[P]C_i[\epsilon:Q]}
		\infer1[choice (where $i=1$ or $2$)]{[P]C_1+C_2[\epsilon:Q]}
	\end{prooftree}
\end{center}
Zhang and Kaminski~\cite{zhang22} supplemented the word ``total'' to the name of this triple, the reason being that the strongest postcondition transformer only include final states that are reachable, dual to the total correctness triple that only allows states that lead to termination. 

\paragraph{Partial Incorrectness}
As the name indicates, when there is ``total incorrectness'', there should be ``partial incorrectness''. 
The red implications in \autoref{fig:triples} correspond to the triples for partial incorrectness. 
As mentioned in the previous section, the slp transformer lacks attention, starting with a sound definition with loops~\cite{wulandari2020VerifyingGraphPrograms,li2011NonlinearMathematicsUncertainty}. 
As far as our knowledge, the underapproximation triple of slp 
$$F\implies slp.C.G$$ 
still needs proper investigation. 
However, the Galois connection relates it to the overapproximation triple of wp: 
$$wp.C.F\implies G$$
Cousot et al.~\cite{cousot13} studied an implication similar to this triple to help lighten the burden of providing specifications in design by contract programming schemes. 
They call it the ``necessary preconditions'', in the sense that these preconditions can not guarantee correctness, but once violated, definitely lead to erroneous final states. 
They achieve this by finding preconditions that will definitely lead to erroneous final states and taking their negation. 
Because they separated the issue of termination from their necessary precondition reasoning by assuming that all programs terminate, their necessary precondition is not exactly an overapproximation of wp. 
Nevertheless, it gives intuition on one of the ways to approach the necessary liberal precondition. 

\paragraph{Necessary Liberal Precondition}
The overapproximation of wlp is the main focus of this thesis, denoted with orange label in \autoref{fig:wp-sp-relation}: 
$$wlp.C.F\implies G$$
The name is decided from the characteristic of this triple: 
\imptt{Necessary}, because its violation will lead to violation of the given postcondition, the definition of a necessary condition in logical terms. 
\imptt{Liberal}, because we take a liberal view on non-termination. We accept initial states that can lead to non-termination as satisfying our precondition. 
\imptt{Precondition}, because we are concerned with the conditions before the execution of a program. 

It reflexes an extremely relaxed view: in this triple, $G$ can be satisfied by all possible initial states, under whose control the program can diverge, successfully terminate, erroneously terminate \dots
The only certainty is that $G$ is the necessary condition, in the sense that once violated, the program can definitely reach erroneous final states. 
We will study this triple further in the upcoming sections. 

\paragraph{Remaining Triple}
The remaining triple denoted with label ``what?'' still awaits investigation as far as we know: 
$$slp.C.G\implies F$$
We think that one approach is to first establish a sound definition for slp, then find a connection between $slp$ and $sp$, either by conjugation or reachability depending on the choice how to resolve non-determinism. 
From here, one can start investigating this triple following similar methods for other triples. 

With this overview of triples studied in literature, we know that the triple $wlp.C.F\implies G$ is still an open question, and not automatically solved by a Galois connection to another implication. 
As a result, we investigate the overapproximation we call the necessary liberal precondition following methods used with other triples by first discussing the possible scenarios of this overapproximation, then look into examples to shed light on the meaning of this implication. 


% The implications noted in black correspond to total correctness~\cite{manna74} $G\implies wp_a.C.F$ and total incorrectness $F\implies sp_a.C.G$~\cite{ohearn2020IncorrectnessLogic} triples, respectively. 
% The blue ones are the partial correctness triples $G\implies wlp_d.C.F$~\cite{hoare69} as well as $sp_a.C.G\implies F$ via Galois connection. 
% The red triples are the partial correctness triples $ F\implies slp_c.C.G$~\cite{zhang22} iff $wp_a.C.F\implies G$ that still require further investigation. 
% The orange triple on the left is the triple found by the necessary liberal precondition and the center of our interest in this thesis, whereas the orange triple on the right is still an unanswered question as far as our knowledge. 

