%!TEX root = ../main-anran-ma.tex 
% so I can build in this tex file too. 
%************************************************
\chapter{Conclusions}\label{ch:conclusion} % $\mathbb{ZNR}$
%************************************************

\section{Conclusions}
In this thesis, I study the weakest liberal precondition transformer and its over-approximation: $G$ such that $wlp.C.F\implies G$. 
I first discuss the definitions of while-loops in its original form~\cite{dijkstra75} and a variant using fixed points. 
I establish an equivalence between the two forms of definitions, validating the prudence of the second version. 
Subsequently, I investigate the $G$ in question. 
Coincidentally, supplementing $G$ with extra restraints using the strongest postcondition transformer, $G$ coincides with the weakest liberal precondition transformer with angelic non-determinism: 
$$(sp.C.\neg G {\implies} \neg F) \wedge
(P{\implies} G \implies \neg(sp.C.P {\implies} \neg F) )
\implies\ G = wlp_a.C.F$$

However, without extra constraints, $G$ can be a precondition where all executions are possible. 
The only certainty is that \hoare{\neg G}{C}{\neg F} is a valid Hoare Triple. 
Regardless, $G$ still finds its usefulness while trying to identify preconditions that lead to erroneous final states, but without sufficient knowledge of all the undesired final states. 
One first finds the weakest liberal precondition with respect to the known ``bad'' final states, then over-approximate the found precondition by ``guessing'' more possible unwanted final states. 

\section{Future Work}
Despite taking inspiration from the incorrectness logic, the methods used in this thesis and the results obtained thereafter are admittedly immature.
The intricate distinction between successful and erroneous final states was not mirrored in this thesis. %, but rather general hand-waving suggestions. 
It would be valuable to study the subject of this thesis with a more sophisticated palette to develop more comprehensive proof rules. 

Additionally, this thesis is only concerned with binary predicates, i.e. a predicate that evaluates to either $true$ or $false$. 
Albeit classic, it might be more interesting to examine the above results in a quantitative setting, where predicates evaluate to more than $true$ or $false$. 
In a quantitative setting, the notion of angelic or demonic non-determinism might be extreme. 
Instead of regarding the non-determinism as completely in or against our favor, which are strong assumptions, what are the implications when the non-determinism resolves partially in or against our favor? 
As the poet Qu Yuan said: 
\CJK{UTF8}{gbsn}
\begin{center}
    路漫漫其修远兮,\\
    \textit{Long, long had been my road and far, far was the journey;} \\ 
    \ \ \,吾将上下而求索。\\
    \textit{I would go up and down to seek my heart’s desire.}~\cite{hawkes2012}
\end{center}




%*****************************************
%*****************************************
%*****************************************
%*****************************************
%*****************************************
