%!TEX root = ../main-anran-ma.tex 
% so I can build in this tex file too. 
%*******************************************************
% Abstract
%*******************************************************
%\renewcommand{\abstractname}{Abstract}
\pdfbookmark[1]{Abstract}{Abstract}
\begingroup
\let\clearpage\relax
\let\cleardoublepage\relax
\let\cleardoublepage\relax

\chapter*{Abstract}
This thesis investigates the necessary liberal precondition, an overapproximation of Dijkstra's weakest liberal precondition (wlp) transformer. 
A discussion of the semantics of wlp and its relatives hints at a scenario where the necessary liberal precondition is useful: 
When the entirety of undesirable postconditions is unknown, by further relaxing wlp, the programmer ends up with a precondition that is likely to lead the program to known or unknown undesired final states upon termination. 

\vfill

\begin{otherlanguage}{ngerman}
\pdfbookmark[1]{Zusammenfassung}{Zusammenfassung}
\chapter*{Zusammenfassung}
Diese Arbeit untersucht die notwendige liberale Vorbedingung, eine Überannäherung an Dijkstras schwächsten liberalen Vorbedingungstransformator (wlp).
Eine Diskussion der Semantik von wlp und seinen Verwandten deutet auf ein Szenario hin, in dem die notwendige liberale Vorbedingung nützlich ist:
Wenn die Gesamtheit der unerwünschten Nachbedingungen unbekannt ist, erhält der Programmierer durch weitere Lockerung von wlp eine Vorbedingung, die das Programm bei Beendigung wahrscheinlich in bekannte oder unbekannte unerwünschte Endzustände führt.
\end{otherlanguage}

\vfill

% \begin{otherlanguage}{ngerman}
% \pdfbookmark[1]{Zusammenfassung}{Zusammenfassung}

\CJK{UTF8}{gbsn}
\chapter*{摘要}
本论文研究了必要自由前提条件,即 Dijkstra 的最弱自由前提条件 (wlp) 的涵盖近似。
首先讨论 wlp 及其近似方法的语义,这暗示了一个必要自由前提条件有用的场景:
当我们不知道所有不好的后置条件时,程序员可以通过进一步放松 wlp来找到一个前提条件,这个条件很可能会导致程序在终止时进入不好的最终状态,不论这些状态是已知或未知的。

% \end{otherlanguage}

\vfill

\endgroup